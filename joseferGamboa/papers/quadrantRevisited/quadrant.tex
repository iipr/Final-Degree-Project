\documentclass[11pt,a4paper]{amsart}

% OTHER PACKAGES USED 
\usepackage{amssymb,amsmath}
\usepackage{amsthm}
\usepackage{latexsym}
\usepackage{graphics}
\usepackage{euscript}
\usepackage[dvips]{graphicx}
\usepackage{pstricks}
%\usepackage[notref,notcite]{showkeys}
\usepackage{multirow}

% END OF OTHER PACKAGES

\hoffset=-13mm
\setlength{\textwidth}{15.5cm}
\setlength{\textheight}{21cm}

\parskip=0.5ex


% NEW ENVIRONMENTS, THEOREMS, COMMANDS

\newenvironment{prooftheorem}[1]%
	{\par\noindent{\it Proof of Theorem \em\ref{#1}\em.}\nopagebreak\normalsize}% 
	{\hfill\linebreak[2]\hspace*{\fill}$\square$\$$5pt]}
	
%\newenvironment{stepsn}[1]{%
%\refstepcounter{theorem}\noindent{\bf (\thetheorem)\quad {#1}}\ }%

\newtheorem{theor}{Theorem}[section]
\newtheorem{prop}[theor]{Proposition}
\newtheorem{cor}[theor]{Corollary}
\newtheorem{lem}[theor]{Lemma}

\theoremstyle{definition}
\newtheorem{define}[theor]{Definition}

\theoremstyle{remark}
\newtheorem{remark}[theor]{Remark}
\newtheorem{remarks}[theor]{Remarks}

\newcommand{\fin}{\hspace*{\fill}$\square$\\[5pt]}
\newcommand{\fina}{\hspace*{\fill}$\square$}
\newcommand{\finsmall}{\hspace*{\fill}{\tiny$\displaystyle\square$}}

\newenvironment{steps}[1]{%
\refstepcounter{theor}\noindent{\bf (\thetheor)\ {#1}}\ }%
{\em}

\newenvironment{stepsb}{%
\refstepcounter{theor}\noindent{\bf (\thetheor)\ }\ }%
{\em}

\newcounter{substep}
\def\thesubstep{\arabic{substep}}

\newenvironment{substeps}[1]{%
\refstepcounter{substep}\noindent{ (\ref{#1}.\thesubstep)\ }\ }%
{\em}

%%%%% Bbb symbols

\newcommand{\K}{{\mathbb K}}  \newcommand{\N}{{\mathbb N}}
\newcommand{\Z}{{\mathbb Z}}  \newcommand{\R}{{\mathbb R}}
\newcommand{\Q}{{\mathbb Q}}  \newcommand{\C}{{\mathbb C}}
\newcommand{\A}{{\mathbb A}}  \newcommand{\F}{{\mathbb F}}
\newcommand{\HH}{{\mathbb H}}  \newcommand{\Bb}{{\mathbb B}}
\newcommand{\sph}{{\mathbb S}} \newcommand{\D}{{\mathbb D}}
\newcommand{\E}{{\mathbb E}}  \newcommand{\PP}{{\mathbb P}}


%%%%% Sheaf symbols

\newcommand{\pol}{{\EuScript K}}
\newcommand{\trian}{{\EuScript T}}
\newcommand{\Qq}{{\EuScript Q}}
\newcommand{\ball}{{\EuScript B}}
\newcommand{\cube}{{\EuScript C}}
\newcommand{\halfplane}{{\EuScript H}}
\newcommand{\squarea}{{\EuScript S}}

%%%%% Operator names

\newcommand{\Int}{\operatorname{Int}}
\newcommand{\Cl}{\operatorname{Cl}}
% \newcommand{\p}{\operatorname{p}}
\newcommand{\rr}{\operatorname{r}}
\newcommand{\TF}{\operatorname{TF}}
\newcommand{\TS}{\operatorname{TS}}
\newcommand{\cl}{\operatorname{Cl}}
\newcommand{\dist}{\operatorname{dist}}
\newcommand{\Id}{\operatorname{Id}}
\newcommand{\im}{\operatorname{im}}
\newcommand{\size}{\mathfrak{s}}

%%%%% Typewriter symbols

\newcommand{\x}{{\tt x}} \newcommand{\y}{{\tt y}}
\newcommand{\z}{{\tt z}} \renewcommand{\t}{{\tt t}}
\newcommand{\s}{{\tt s}}

%%%%% Varia

\newcommand{\ol}{\overline}
\newcommand{\colonb}{{:}\;}
\newcommand{\qq}[1]{\langle{#1}\rangle}

% END OF NEW ENVIRONMENTS, THEOREMS, COMMANDS

\title{On the open quadrant as a polynomial image of $\R^2$ (revisited)}

% author one & two information
\author{Jos\'e F. Fernando}
\author{J.M. Gamboa}
\address{Departamento de \'Algebra, Facultad de Ciencias Matem\'aticas, Universidad Complutense de Madrid, 28040 MADRID (SPAIN)}
\curraddr{}
\email{josefer@mat.ucm.es, jmgamboa@mat.ucm.es}
\thanks{Authors supported by Spanish GAAR MTM2008-00272. First and second authors also supported by Proyecto Santander Complutense PR34/07-15813 and GAAR Grupos UCM 910444.\\
\indent This article is based on a part of the doctoral dissertation of the third author.}

% author three information

\author{Carlos Ueno}
\address{Departamento de Matem\'aticas, IES Jand\'\i a, Morro del Jable, Fuerteventura
35625 LAS PALMAS (SPAIN).
}
\email{carlos.ueno@terra.es}

\begin{document}

\begin{abstract}
We present a new proof of the fact that the (first) open quadrant of $\R^2$ is a polynomial image of $\R^2$, using a new polynomial map different to the one proposed in \cite{fg1}. As it is natural we compare several characteristics of both polynomial maps. As a byproduct we see that the exterior of the $n$-dimensional cube and the exterior of the $n$-dimensional open ball of $\R^n$ are polynomial images of $\R^n$.
\end{abstract}
%  \date is required; it is the date received by the editor. 
\date{}
\subjclass[2000]{Primary 14P10; Secondary 14Q99}
\keywords{Open quadrant, polynomial map, polynomial image, size of a polynomial, $n$-dimensional cube, $n$-dimensional open ball.}
\maketitle

% THE ARTICLE STARTS HERE 

% SECTION: INTRODUCTION
\section{Introduction}\label{s1}

A map $f=(f_1,\ldots,f_n):\R^m\rightarrow\R^n$ is said to be \em polynomial \em if each component $f_i\in\R[\x_1,\ldots,\x_m]$ is a polynomial in $m$ variables and coefficients in $\R$. The well-known Tarski-Seidenberg's Theorem (\cite[1.4]{bcr}) states that the image of any polynomial map $f:\R^m\rightarrow \R^n$ is a semialgebraic subset of $\R^n$, i.e. it can be written as a finite boolean combination of sets described using polynomial equalities and inequalities. In an \em Oberwolfach \em week (\cite{g}), the second author proposed to characterize the semialgebraic sets of $\R^n$ which are polynomial images of $\R^m$. In fact, as far as we know, \cite{fg1} is the first article concerning this subject, and there the first and second authors presented some general geometrical and topological properties of those semialgebraic subsets $S\subset\R^n$ which are polynomial images of $\R^m$. Moreover, in the same article the authors provided a proof of the following result:

\begin{theor}\label{mainthm}
The open quadrant $\Qq=\{(x,y)\in\R^2:\ x>0,\ y>0\}$ is a polynomial image of $\R^2$.
\end{theor}

Once this is proved it is straightforward to show (see \cite[2.6]{fg1}) the following generalization for $\R^n$:

\begin{cor}\label{mainthm1}
Let $n\geq 2$ and let $\ell_1,\ldots,\ell_r$ be independent linear forms in $n$ variables. Then, $S=\{x\in\R^n:\ \ell_1(x)>0,\ldots,\ell_r(x)>0\}$ is a polynomial image ~of $\R^n$.
\end{cor}

The proof of \ref{mainthm} proposed in \cite{fg1} runs as follows. First, the properties that must enjoy a couple of polynomials $F_1,F_2\in\R[\x,\y]$ such that the image of the map $F=(F_1,F_2):\R^2\to\R^2$ is $\Qq$ are carefully discussed. Taking such properties into account it is constructed an explicit polynomial map whose image satisfies such requirements, and so it seems to be a good candidate to have the quadrant $\Qq$ as image. Finally, after cumbersome computations that involved the use of packages performing the Sturm sequence of polynomials of high degree, it is proved that in fact the image of $F$ is $\Qq$. 

The purpose of this article is to present a new proof of \ref{mainthm}, using a polynomial map $P:\R^2\to\R^2$ for which the checking of the equality $P(\R^2)=\Qq$ is much more tricky than the one for the polynomial map proposed in \cite{fg1}, but it does not involve computations which should be performed by a computer. 

The strategy developed here is quite different to the one presented in \cite{fg1}. While there the initial efforts are devoted to construct a ``winner'' polynomial map, here we approach the problem in several steps and we obtain the ``winner'' polynomial map by composing six intermediate polynomial maps whose images can be either computed or estimated (depending on our requirements) ``by hand" using elementary methods. To that end we construct a finite sequence of polynomial maps $F_i:\R^2\to\R^2$ for $1\leq i\leq 6$ such that $F=F_6\circ F_5\circ F_4\circ F_3\circ F_2\circ F_1$ maps $\R^2$ onto $\Qq$. This way, instead of trying to solve the problem just in one step and then crashing into the difficult task of computing the image of the constructed polynomial map, in our new proof we propose a sequence of steps such that each image $F_{i+1}\big(F_i(\R^2)\big)$ (resp. $F_1(\R^2)$) is easily determined. This kind of strategy has already been successfully used in \cite{fgu1}, \cite{u1} and \cite{u2} to approach some of the questions proposed in \cite{fg2} and also to prove that the complements of convex polyhedra in $\R^n$ for $n=2,3$ are polynomial images of $\R^n$. 

\vspace{2mm}
\begin{stepsb}\label{size}
As one can expect, we have compared the \em sizes \em of the components of the ``new'' polynomial map having the open quadrant $\Qq$ as its image, with the \em sizes \em of the components of the ``old'' map already proposed in \cite{fg1}. Given a polynomial $P\in\R[\x_1,\ldots,\x_n]\setminus\{0\}$ we consider the following numbers:
\begin{itemize}
\item[(i)] ${\mathfrak n}_k(P)=$ the number of monomials of $P$ of degree $k$ for each $k\geq0$.
\item[(ii)] ${\mathfrak m}(P)=$ the maximal size $m$ (number of digits) of a nonzero coefficient of $P$.
\item[(iii)] ${\mathfrak c}_k(P)=$ the number of coefficients of $P$ of size $k$ for each $k\geq 1$.
\item[(iv)] $\deg(P)=\max\{k\geq0:\ {\mathfrak n}_k(P)>0\}$\ =\ total degree of $P$. 
\end{itemize} 
We define the \em size \em of $P$ as the $5$-tuple of nonnegative integers
$$
\size(P)=\Big(\sum_{k=1}^{\deg(P)}(k+1){\mathfrak n}_k(P),\sum_{k=1}^{{\mathfrak m}(P)}k{\mathfrak c}_k(P),\sum_{k=1}^{\deg(P)}{\mathfrak n}_k(P)=\sum_{k=1}^{{\mathfrak m}(P)}{\mathfrak c}_k(P), \deg(P),{\mathfrak m}(P)\Big).
$$
Given two polynomials $P,Q\in\R[\x_1,\ldots,\x_n]\setminus\{0\}$, we compare their sizes lexicographically using the usual ordering of the integers. The size of a polynomial is a tentative to quantify the effect (in the increasing of number of monomials, coefficients, degree, etc) that will produce the composition of a polynomial map with a polynomial. The size of a polynomial measures the number of monomials in a pondered way (the effect of a monomial of a higher degree is larger than the effect of a monomial of lower degree), the total number of digits of the coefficients of the polynomial, the number of monomials (not pondered), its degree and the maximal size, that is, number of digits, of a nonzero coefficient of $P$. Of course, many other variables could be chosen to define the size, but we consider that the ones chosen are significative enough. 
\end{stepsb}

\vspace{2mm}
As we explained in \cite{fg2}, the optimization of polynomial functions $G$ on semialgebraic sets $S\subset\R^n$ is much easier in case $S=F(\R^m)$ is a polynomial image of some $\R^m$, because in this case the problem is equivalent to optimize on $\R^m$ the composition $G\circ F$.

The effect of the composition on the optimization procedure can be quantified by comparing the sizes $\size(G)$ and $\size(G\circ F)$. Nevertheless, an initial test of this effect is the computation of the sizes of the components of the polynomial map $F$. This is what we do for the ``old'' and ``new'' maps producing the quadrant as a polynomial image. As we will see, the size of one of the components of the new map is rather larger than the size of the components of the ``old'' map, which suggests that for computational purposes the ``old'' map could be better. However, as we have already mentioned, for theoretical purposes the ``new'' map is better with no discussion.

We finish this article with some applications of \ref{mainthm}. As it has being already pointed out in \cite{fgu1}, the techniques introduced there to prove that the complement of a polyhedron $\pol\subset\R^3$ is a polynomial image of $\R^3$ have been developed \em ad hoc \em for $\R^3$ and, as we have already commented there, it seems rather difficult to generalize them for complements of polyhedra of $\R^n$. However, using \ref{mainthm} and the special symmetry of the $n$-dimensional cube, we can modify those arguments to prove that the complement of the $n$-dimensional cube is a polynomial image of $\R^n$ (see \ref{n2} and \ref{box}). 

As we have already observed in \cite{fg2}, for a semialgebraic set $S\subset\R^n$, the least integer ${\rm p}(S)=p$ such that $S$ is the image of a polynomial map $\R^p\to\R^n$ is greater than or equal to the dimension of $S$. Thus, if $\cube_n\subset\R^n$ is an $n$-dimensional cube, then ${\rm p}(\R^n\setminus\cube_n)=n$. Using this fact, we prove that the complement of an $n$-dimensional open ball $\ball$ is a polynomial image of $\R^n$, an so that ${\rm p}(\R^n\setminus\ball)=n$ (see \ref{openball}). This result answers a question already proposed in \cite[7.3.3]{fg2}. On the other hand, as we point out in Section \ref{s4}, the complement of the $n$-dimensional closed ball $\ol{\ball}$ is not a polynomial image of $\R^n$ but it is a polynomial image of $\R^{n+1}$, hence ${\rm p}(\R^n\setminus{\ol{\ball}})=n+1$.
 
The article is organized as follows. In Section \ref{s2}, we present the alternative proof of \ref{mainthm}. In Section \ref{s3}, we compare the sizes of the ``old'' and the ``new'' polynomial maps having the open quadrant as image. Finally, we present in Section \ref{s4} the already mentioned results concerning complements of $n$-dimensional cubes and $n$-dimensional balls.

\section{The open quadrant as a polynomial image of $\R^2$}\label{s2}

To make the proof of \ref{mainthm} nimbler and more comprehensible, we prove before some preliminary results which will be understood ``a posteriori'' as steps of the proof of \ref{mainthm}. For a better understanding of the strategy employed in the proof of \ref{mainthm} we have extracted from it, as auxiliary lemmas, the computation of the images of some semialgebraic sets under polynomial maps of low degree, which are in fact some of the polynomial maps already mentioned in the introduction, whose composition produces the open quadrant as a polynomial image of $\R^2$.

We begin by including, for the sake of completeness, a straightforward proof of the already known fact that the upper half plane $\halfplane=\{(x,y)\in\R^2:\ y>0\}$ is a polynomial image of $\R^2$ (see also \cite[1.4(iv)]{fg1}, \cite{u1}).

\begin{lem}\label{step1}
The image of the polynomial map 
$$
F=(F_1,F_2):\R^2\to\R^2,\ (x,y)\mapsto(y(xy-1),(xy-1)^2+x^2)
$$ 
is the upper half plane $\halfplane=\{(x,y)\in\R^2:\ y>0\}$.
\end{lem}
\begin{proof}
Indeed, since the polynomial $F_2(\x,\y)=(\x\y-1)^2+\x^2$ is strictly positive on $\R^2$, it follows that $F(\R^2)\subset\halfplane$. Thus, we have to check that each point $(a,b)\in\R^2$ with $b>0$ belongs to the image of $F$ or, equivalently, that for $a\in\R$ and $b>0$ the plane curves 
$$
C_a=\{(x,y)\in\R^2:\ y(xy-1)-a=0\}\quad\text{and}\quad\Gamma_b=\{(x,y)\in\R^2:\ (xy-1)^2+x^2-b=0\}
$$ 
have nonempty intersection. Observe that $(x,y)\in(C_a\cap\Gamma_b)\setminus\{y=0\}$ if and only if $y$ is a nonzero root of the polynomial 
$$
P(\y)=\y^4\left(b-F_2\Big(\frac{a+\y}{\y^2},\y\Big)\right)=b\y^4-(1+a^2)\y^2-2a\y-a^2.
$$
Observe that for $a\neq0$, $P(0)<0$ whereas $\lim_{y\to+\infty}P(y)=+\infty$. Thus, the polynomial $P$ has a positive real root $y_{(a,b)}$ and so 
$F\Big(\frac{a+y_{(a,b)}}{y_{(a,b)}^2},y_{(a,b)}\Big)=(a,b)$. 

On the other hand, if $a=0$ and $b>0$, a straightforward computation shows that $F(\sqrt{b},1/\sqrt{b})=(0,b)$, and we are done.
\end{proof}

\begin{lem}\label{step2}
The image $S\subset\R^2$ of the semialgebraic set $\Qq'=\{(x,y)\in\R^2:\ x\geq0,\ y>0\}$ under the polynomial map $F=(F_1,F_2):\R^2\to\R^2,\ (x,y)\mapsto\big(x+y(xy-1)^2,y\big)$ satisfies the following properties:
\begin{itemize}
\item[(i)] $S\subset\Qq=\{(x,y)\in\R^2:\ x>0,y>0\}$.
\item[(ii)] $A=\{(x,y)\in\R^2:\ x>0,\ xy-1\geq0\}\subset S$.
\item[(iii)] $B=\{(x,y)\in\R^2:\ 0<y\leq x\}\subset S$.
\end{itemize}
\end{lem}
\begin{proof}
Indeed, property (i) is clear because the restriction $F_1|_{\Qq'}$ is strictly positive. Next, for each $y>0$ consider the set $A_y=\{x\in\R:\ (x,y)\in A\}=[1/y,+\infty)$ and observe that $A=\bigcup_{y>0}A_y^*$, where $A_y^*=A_y\times\{y\}$. Note that $F_1(A_y^*)$ is an interval containing $A_y$. This is so because $\lim_{x\to+\infty}x+y(xy-1)^2=+\infty$ and $F_1(1/y,y)=1/y$ for all $y>0$. Hence $A_y^*\subset F(A_y^*)$, and since $A\subset\Qq'$, 
$$
A=\bigcup_{y>0}A_y^*\subset\bigcup_{y>0}F(A_y^*)=F\Big(\bigcup_{y>0}A_y^*\Big)=F(A)\subset F(\Qq')=S,
$$
and (ii) holds. For (iii) we write $B_y=\{x\in\R:\ (x,y)\in B\}=[y,+\infty)$ and observe that $B=\bigcup_{y>0}B_y^*$ where $B_y^*=B_y\times\{y\}$. Moreover, since $F_1(0,y)=y$ and $\lim_{x\to+\infty}x+y(xy-1)^2=+\infty$, we deduce that $F_1\big([0,+\infty)\times\{y\}\big)$ is an interval that contains $B_y$ for all $y>0$. Thus, since $\Qq'=\bigcup_{y>0}[0,+\infty)\times\{y\}$ and $B_y^*\subset F\big([0,+\infty)\times\{y\}\big)$ it follows that
$$
B=\bigcup_{y>0}B_y^*\subset\bigcup_{y>0}F([0,+\infty)\times\{y\})=F\Big(\bigcup_{y>0}[0,+\infty)\times\{y\}\Big)=F(\Qq')=S,
$$
and (iii) holds, as wanted.
\end{proof}

\begin{lem}\label{step3}
Let $A=\{(x,y)\in\R^2:\, x>0, \ xy-1\geq0\}$, $B=\{(x,y)\in\R^2:\, 0<y\leq x\}$ and $C=A\setminus B$. 
Consider the linear map $F:\R^2\to\R^2,\, (x,y)\mapsto(x,y-x)$. Then, $F(B)=B'$, where $B'=\{(x,y)\in\R^2:\ -x<y\leq0\}$, and $A\subset F(C)$.
\end{lem}
\begin{proof}
First, observe that for a fixed $x$ the second coordinate of $F$ is increasing with respect to $y$. If we set $B_x=\{y\in\R:\ 0<y\le x\}$ and 
$B'_x=\{y\in\mathbb R:\ -x<y\le 0\}$, then $B=\bigcup_{x>0}\{x\}\times B_x$, $B'=\bigcup_{x>0}\{x\}\times B'_x$, and since $F(\{x\}\times B_x)=\{x\}\times B'_x$ for 
any $x>0$, we conclude that $F(B)=B'$.

Now, we show that $A\subset F(C)$. Let us set for $x>0$
$$
A_x=\{y\in\mathbb R:\, xy-1\ge 0\}=\{y\in\mathbb R:\, y\ge 1/x\}\quad\text{and}\quad C_x=A_x\setminus B_x.
$$
Notice that $C=\bigcup_{x>0}\{x\}\times C_x$ and 
\begin{multline*}
F(\{x\}\times C_x)=\{F(x,y):\, y>x,\, y\ge 1/x\}=\{(x,y-x):\, y>x,\, y\ge 1/x\}=\\
\{(x,y'):\, y'>0,\, y'\ge (1-x^2)/x\}\supset\{(x,y'):\, y'\ge 1/x\}=\{x\}\times A_x.
\end{multline*}
Thus, $F(\{x\}\times C_x)\supset \{x\}\times A_x$ for all $x>0$ and therefore $F(C)\supset A$.
\end{proof}

\begin{lem}\label{step4}
Let $D=\{(x,y)\in\R^2:\ -x<y\leq0\}$ and let $E\subset\R^2$ be such that 
$$
A=\{(x,y)\in\R^2:\ xy-1\geq0,x>0\}\subset E\subset\Qq''=\{(x,y)\in\R^2:\ x>0,y\geq0\}.
$$ 
Consider the polynomial map $F=(F_1,F_2):\R^2\to\R^2,\ (x,y)\mapsto(x,y(2-xy)^2)$. Then, 
$$
F(D\cup E)=\{(x,y)\in\R^2:\ x>0,y>-x(2+x^2)^2\}.
$$
\end{lem}
\begin{proof}
First, let us prove that $F(A)=\Qq''$. Indeed, set $A_x=\{y\in\R:\ (x,y)\in A\}=[1/x,+\infty)$ and observe that $A=\bigcup_{x>0}\{x\}\times A_x$. Observe that the restriction $F_2|_{A_x}$ is positive semidefinite, $F_2(x,2/x)=0$ and $\lim_{y\to+\infty}F_2(x,y)=+\infty$ for each $x>0$. Thus, since the image of $F_2|_{\{x\}\times A_x}$ is an interval, we conclude that $F_2\big(\{x\}\times A_x\big)=[0,+\infty)$. Hence, $F\big(\{x\}\times A_x\big)=\{x\}\times[0,\infty)$ and so 
$$
F(A)=\bigcup_{x>0}F\big(\{x\}\times A_x\big)=\bigcup_{x>0}\{x\}\times[0,\infty)=\Qq''.
$$
Next, observe that since $A\subset E\subset\Qq''$ and $F(\Qq'')\subset\Qq''$, 
$$
\Qq''=F(A)\subset F(E)\subset F(\Qq'')\subset\Qq'',
$$
and so $F(E)=\Qq''$. To finish, let us show that 
$$
F(D)=\{(x,y)\in\R^2:\ x>0,-x(2+x^2)^2<y\leq0\}.\qquad(\ast) 
$$
Once this is done, we have 
\begin{multline*}
F(D\cup E)=F(D)\cup F(E)=\Qq''\cup\{(x,y)\in\R^2:\ x>0,\ -x(2+x^2)^2<y\leq0\}\\
=\{(x,y)\in\R^2:\ x>0,\ y>-x(2+x^2)^2\}.
\end{multline*}

Thus, let us proceed to show $(\ast)$. Observe first that the partial derivative
$$
\frac{\partial F_2}{\partial \y}=(2-\x\y)^2+2(2-\x\y)(-\x\y)=4-8\x\y+3\x^2\y^2
$$
is strictly positive on the set $\{(x,y)\in\R^2:\ -xy\geq0\}\supset D'=\{(x,y)\in\R^2:\ -x\leq y\leq0\}$. Thus, the restriction $F_2|_{\{x\}\times[-x,0]}$ is strictly increasing for each $x>0$. Hence, 
$$
F_2\big(\{x\}\times(-x,0]\big)=\big(F_2(x,-x),F_2(x,0)\big]=\big(-x(2+x^2)^2,0\big].
$$
Therefore,
\begin{equation*}
\begin{split}
F(D)&=F\Big(\bigcup_{x>0}\{x\}\times(-x,0]\Big)=\bigcup_{x>0}\{x\}\times F_2\big(\{x\}\times(-x,0]\big)\\
&=\bigcup_{x>0}\{x\}\times\big(-x(2+x^2)^2,0\big]=\{(x,y)\in\R^2:\ x>0,-x(2+x^2)^2<y\leq0\},
\end{split}
\end{equation*}
and we are done.
\end{proof}


\begin{lem}\label{step5}
Let $C=\{(x,y)\in\R^2:\ x>0, \ y>p(x)\}$, where $p\in\R[\x]$ is a polynomial. Then, the open cuadrant $\Qq=\{(x,y)\in\R^2:\ x>0, \ y>0\}$ is the image of $C$ under the polynomial map $F=(F_1,F_2):\R^2\to\R^2,\ (x,y)\mapsto(x,y-p(x))$.
\end{lem}
\begin{proof}
Observe first that $C=\bigcup_{x>0}\{x\}\times C_x$ where $C_x=\{y\in\R:\ y>p(x)\}=\big(p(x),+\infty\big]$, and $\Qq=\bigcup_{x>0}\{x\}\times\Qq_x$ where $\Qq_x=\{y\in\R:\ y>0\}=(0,+\infty)$. Moreover, $F_2\big(\{x\}\times C_x\big)=\Qq_x$ and so $F(C)=\bigcup_{x>0}\{x\}\times F_2\big(\{x\}\times C_x\big)=\bigcup_{x>0}\{x\}\times \Qq_x=\Qq$.
\end{proof}

Now, we are ready to prove Theorem \ref{mainthm}.

\begin{proof}[Proof of Theorem \em\ref{mainthm}]
First, by \ref{step1}, the image of the polynomial map 
$$
F_1:\R^2\to\R^2,\ (x,y)\mapsto(x_1,y_1)=\big(y(xy-1),(xy-1)^2+x^2\big)
$$ 
is the upper half-plane $\halfplane=\{(x_1,y_1)\in\R^2:\ y_1>0\}$. Next, consider the polynomial map 
$$
F_2:\R^2\to\R^2,\ (x_1,y_1)\mapsto(x_2,y_2)=(x_1^2,y_1),
$$ 
and observe that $F_2(\halfplane)=\Qq'=\{(x_2,y_2)\in\R^2:\ x_2\geq0,y_2>0\}$. Now, we want to get rid of the half-line $\{(x_2,y_2)\in\R^2:\ x_2=0,\ y_2>0\}$, and we consider the polynomial map 
$$
F_3:\R^2\to\R^2,\ (x_2,y_2)\mapsto(x_3,y_3)=\big(x_2+y_2(x_2y_2-1)^2,y_2\big).
$$
Denote $A=\{(x_3,y_3)\in\R^2:\ x_3>0,\ x_3y_3-1\geq0\}$ and $B=\{(x_3,y_3)\in\R^2:\ 0<y_3\leq x_3\}$. By \ref{step2} we know that 
$$
A\cup B\subset S=F_3(\Qq')\subset \Qq=\{(x_3,y_3)\in\R^2:\ x_3>0,\ y_3>0\}.
$$ 
Next, consider the linear map
$$
F_4:\R^2\to\R^2,\ (x_3,y_3)\mapsto(x_4,y_4)=(x_3,y_3-x_3),
$$
and observe that if $C=A\setminus B$, $D=\{(x_4,y_4)\in\R^2:\ -x_4<y_4\leq0\}$ and $T=F_4(S)$, we have, ~by \ref{step3}, 
\begin{multline*}
A\cup D\subset F_4(C)\cup F_4(B)=F_4(C\cup B)=F_4(A\cup B)\\
\subset T\subset F_4(\Qq)=\{(x_4,y_4)\in\R^2:\ x_4>0, \ x_4+y_4>0\}=\Qq\cup D.
\end{multline*}
Observe that the sets $A$ and $D$ are disjoint and so $A\subset E=T\setminus D\subset \Qq$. Consider now the polynomial map
$$
F_5:\R^2\to\R^2,\ (x_4,y_4)\mapsto(x_5,y_5)=\big(x_4,y_4(2-x_4y_4)^2\big).
$$
From \ref{step4} we get the equality
$$
Z=F_5(T)=F_5(E\cup D)=\{(x_5,y_5)\in\R^2:\ x_5>0,\ y_5>-x_5(2+x_5^2)^2\}.
$$
Next, consider the polynomial map
$$
F_6:\R^2\to\R^2,\ (x_5,y_5)\mapsto(x_6,y_6)=(x_5,y_5+x_5(2+x_5^2)^2)
$$
and note that, by \ref{step5}, $F_6(Z)=\Qq$. 

Finally, putting all together we conclude that the image of the polynomial map $P=(P_1,P_2)=F_6\circ F_5\circ F_4\circ F_3\circ F_2\circ F_1:\R^2\to\R^2$ is the open quadrant $\Qq$, and we are done.
\end{proof}

\section{The ``old'' map versus the ``new'' map}\label{s3}

The purpose of this section is to compare the polynomial maps constructed to represent the open quadrant $\Qq=\{(x,y)\in\R^2:\ x>0,y>0\}$, the ``old'' one which was presented in \cite{fg1} and the ``new'' one proposed in the previous section. We begin by analyzing some properties of the ``old map'':

\vspace{2mm}
\begin{steps}{The ``old''map.}\label{old}
In \cite[\S3]{fg1} the first and second authors of this work proved, with quite some effort (of them) and cumbersome computations involving Sturm algorithm (performed with Maple), that the image of the polynomial map $F=(F_1,F_2):\R^2\to\R^2$ whose coordinates are given by the polynomials
\begin{equation*}
\begin{split}
F_1(\x,\y)&=(1-\x^3\y+\y-\x\y^2)^2+(\x^2\y)^2\\
&=\x^6\y^2+\x^4(2\y^3+\y^2)+\x^3(-2\y^2-2\y)+\x^2\y^4+\x(-2\y^3-2\y^2)+\y^2+2\y+1,\\
F_2(\x,\y)&=(1-\x\y+\x-\x^4\y)^2+(\x^2\y)^2\\
&=\x^8\y^2+\x^5(2\y^2-2\y)+\x^4(\y^2-2\y)+\x^2(\y^2-2\y+1)+\x(-2\y+2)+1,
\end{split}
\end{equation*}
is $\Qq\cup\{(1,0),(0,1)\}$ and that $F^{-1}(\{(1,0),(0,1)\})=\{(-1,0),(0,-1)\}$. Let us compute the size of $F_1$ and $F_2$ and the involved numbers ${\mathfrak n}_i,{\mathfrak c}_j$ (see \ref{size}):

\begin{center}
{\Small
\renewcommand*{\arraystretch}{1.5}
\begin{tabular}{|c|c|c|c|c|c|c|c|c|c|c|c|c|}
\hline
\multirow{2}{*}{$F_1$}&${\mathfrak n}_0=1$&${\mathfrak n}_1=1$&${\mathfrak n}_2=1$&${\mathfrak n}_3=1$&${\mathfrak n}_4=2$&${\mathfrak n}_5=1$&${\mathfrak n}_6=2$&${\mathfrak n}_7=1$&${\mathfrak n}_8=1$&${\mathfrak n}_9=0$&${\mathfrak n}_{10}=0$\\
\cline{2-12}
&\multicolumn{4}{|c|}{${\mathfrak c}_1=11$}&\multicolumn{7}{|c|}{${\mathfrak s}=(57,11,11,8,1)$}\\
\hline
\multirow{2}{*}{$F_2$}&${\mathfrak n}_0=1$&${\mathfrak n}_1=1$&${\mathfrak n}_2=2$&${\mathfrak n}_3=1$&${\mathfrak n}_4=1$&${\mathfrak n}_5=1$&${\mathfrak n}_6=2$&${\mathfrak n}_7=1$&${\mathfrak n}_8=0$&${\mathfrak n}_9=0$&${\mathfrak n}_{10}=1$\\
\cline{2-12}
&\multicolumn{4}{|c|}{${\mathfrak c}_1=11$}&\multicolumn{7}{|c|}{${\mathfrak s}=(57,11,11,10,1)$}\\
\hline
\end{tabular}}
\end{center}

Moreover, it is clear that to express $\Qq$ as a polynomial image of $\R^2$ it is enough to compose $F$ with a map $F':\R^2\to\R^2$ whose image is $\R^2\setminus\{(-1,0),(0,-1)\}$. In \cite[\S2]{fg1}, the authors propose a procedure to construct, for any finite subset $S\subset\R^n$, where $n\geq2$, a polynomial map $\R^n\to\R^n$ whose image is the complement of $S$ in $\R^n$. The proof approaches the problem for a general $n\geq2$. However, if $n=2$ and $S$ has exactly two points the proposed construction can be rather simplified as we show in the following result, which is proved using the same ideas of \cite[2.1]{fg1}. Observe first that after an affine change of coordinates $S$ can be assumed to be the set $\{(0,0),(-1,0)\}$.
\end{steps}

\begin{lem}\label{2pt}
Let $r\in\R\setminus\{0,1\}$ and $S=\{(0,0),(-1,0)\}$. Then, $\R^2\setminus S$ is the image of the polynomial map 
$$
H=(H_1,H_2):\R^2\to\R^2,\ (x,y)\mapsto\big(xy-r,x^2(xy-r)(xy-r+1)+y\big).
$$ 
\end{lem}
\begin{proof}
Indeed, denote $p_1=(0,0)$ and $p_2=(-1,0)$. Suppose first that there exists a point $q=(q_1,q_2)\in\R^2$ such that $H(q)=p_\ell$ for some $\ell=1,2$. Then, $H_1(q)=q_1q_2-r$ is either $0$ or $-1$, which implies $q_1^2(q_1q_2-r)(q_1q_2-r+1)=0$. Thus $q_2=H_2(q)=0$ and so $r$ must be equal to $0$ or $1$, a contradiction. Hence, $H(\R^2)\subset\R^2\setminus S$. 

Conversely, let $u=(u_1,u_2)\in\R^2\setminus S$. We have to solve the system of polynomial equations:
$$
\left\{\begin{array}{rcl}
H_1(x)&=&xy-r=u_1,\\[4pt]
H_2(x)&=&x^2(xy-r)(xy-r+1)+y=u_2.
\end{array}\right.
$$
If $u_1=-r$ then $H(0,u_2)=u$. If $u_1\neq-r$, substituting $y=\frac{u_1+r}{x}$ in $H_2$, we see that $x$ must be a nonzero root of the polynomial
$$
P(\t)=u_1(u_1+1)\t^3-u_2\t+u_1+r,
$$
which has odd degree (because $u\not\in S$) and $P(0)=u_1+r\neq 0$. Now, if $q_1$ is a real root of $P$, then $q_1\neq0$ and $H(q_1,\frac{u_1+r}{q_1})=u$, as wanted.
\end{proof}

\begin{stepsb}
After this small digression, we continue the discussion already began in \ref{old}. Let us exhibit here a polynomial map $F'=(F_1',F_2'):\R^2\to\R^2$ whose image is $\R^2\setminus\{(-1,0),(0,-1)\}$. First, consider the affine bijection
$$
Q:\R^2\to\R^2,\ (x,y)\mapsto(x,y-x-1),
$$
which maps $(-1,0)$ to $(-1,0)$ and $(0,0)$ to $(0,-1)$. Now, by \ref{2pt}, the image of the polynomial map $F'=Q\circ H:\R^2\to\R^2$, choosing for instance $r=-1$, is $\R^2\setminus\{(-1,0),(0,-1)\}$, and so the image of the composition $G=F\circ F':\R^2\to\R^2$ is the open quadrant $\Qq$. Let us take a look at the coordinates of the composition:
{\tiny\begin{align*}
G_1(\x,\y)&=\y^{10}\big(\x^{18}+2\x^{16}+\x^{14}\big)+\y^{9}\big(-14\x^{17}-30\x^{15}+4\x^{14}-18\x^{13}+6\x^{12}-2\x^{11}+2\x^{10}\big)\\
&+\y^{8}\big(87\x^{16}+202\x^{14}-44\x^{13}+143\x^{12}-72\x^{11}+34\x^{10}-30\x^{9}+7\x^{8}-2\x^{7}+\x^{6}\big)\\
&+\y^{7}\big(-316\x^{15}-804\x^{13}+208\x^{12}-662\x^{11}+378\x^{10}-226\x^{9}+192\x^{8}-66\x^{7}+26\x^{6}-12\x^{5}+2\x^{4}\big)\\
&+\y^{6}\big(743\x^{14}+2094\x^{12}-552\x^{11}+1985\x^{10}-1134\x^{9}+828\x^{8}-688\x^{7}+269\x^{6}-128\x^{5}+58\x^{4}-12\x^{3}+\x^{2}\big)\\
&+\y^{5}\big(-1182\x^{13}-3726\x^{11}+900\x^{10}-4046\x^{9}+2124\x^{8}-1922\x^{7}+1522\x^{6}-622\x^{5}+340\x^{4}-146\x^{3}+28\x^{2}-2\x\big)\\
&+\y^{4}\big(1289\x^{12}+4582\x^{10}-924\x^{9}+5702\x^{8}-2538\x^{7}+3022\x^{6}-2150\x^{5}+906\x^{4}-558\x^{3}+207\x^{2}-30\x+1\big)\\
&+\y^{3}\big(-952\x^{11}-3840\x^{9}+584\x^{8}-5504\x^{7}+1884\x^{6}-3286\x^{5}+1910\x^{4}-888\x^{3}+586\x^{2}-162\x+12\big)\\
&+\y^{2}\big(456\x^{10}+2096\x^{8}-208\x^{7}+3487\x^{6}-792\x^{5}+2408\x^{4}-978\x^{3}+621\x^{2}-372\x+55\big)\\
&+\y\big(-128^{9}-672^{7}+32^{6}-1308^{5}+144^{4}-1080^{3}+220^{2}-308+112\big)+\big(16\x^{8}+96\x^{6}+220\x^{4}+224\x^{2}+85\big),
\\[4pt]
G_2(\x,\y)&=\y^{12}\x^{16}+\y^{11}\big(-14\x^{15}-2\x^{13}+2\x^{12}\big)+\y^{10}\big(89\x^{14}+26\x^{12}-22\x^{11}+\x^{10}-2\x^{9}+\x^{8}\big)\\
&+\y^{9}\big(-338\x^{13}-152\x^{11}+108\x^{10}-12\x^{9}+20\x^{8}-8\x^{7}\big)+\y^{8}\big(849\x^{12}+524\x^{10}-308\x^{9}+64\x^{8}-88\x^{7}+28\x^{6}\big)\\
&+\y^{7}\big(-1476\x^{11}-1176\x^{9}+558\x^{8}-198\x^{7}+220\x^{6}-54\x^{5}\big)+\y^{6}\big(1808\x^{10}+1792\x^{8}-662\x^{7}+391\x^{6}-340\x^{5}+61\x^{4}\big)\\
&+\y^{5}\big(-1562\x^{9}-1878\x^{7}+514\x^{6}-512\x^{5}+332\x^{4}-40\x^{3}\big)+\y^{4}\big(944\x^{8}+1344\x^{6}-258\x^{5}+447\x^{4}-202\x^{3}+15\x^{2}\big)\\
&+\y^{3}\big(-398\x^{7}-644\x^{5}+86\x^{4}-254\x^{3}+74\x^{2}-4\x\big)+\y^{2}\big(121\x^{6}+206\x^{4}-22\x^{3}+90\x^{2}-18\x+1\big)\\
&+\y\big(-28\x^{5}-48\x^{3}+4\x^{2}-20\x+4\big)+\big(4\x^{4}+8\x^{2}+4\big).
\end{align*}}
Let us compute the sizes of $G_1$ and $G_2$ and the involved numbers ${\mathfrak n}_i,{\mathfrak c}_j$ (see \ref{size}):
\begin{center}
{\Small
\renewcommand*{\arraystretch}{1.5}
\begin{tabular}{|c|c|c|c|c|c|c|c|c|c|c|c|c|}
\hline
\multirow{4}{*}{$G_1$}&${\mathfrak n}_0=1$&${\mathfrak n}_1=1$&${\mathfrak n}_2=3$&${\mathfrak n}_3=3$&${\mathfrak n}_4=5$&${\mathfrak n}_5=4$&${\mathfrak n}_6=6$&${\mathfrak n}_7=5$&${\mathfrak n}_8=7$&${\mathfrak n}_9=5$\\
\cline{2-11}
&${\mathfrak n}_{10}=6$&${\mathfrak n}_{11}=5$&${\mathfrak n}_{12}=6$&${\mathfrak n}_{13}=4$&${\mathfrak n}_{14}=6$&${\mathfrak n}_{15}=4$&${\mathfrak n}_{16}=5$&${\mathfrak n}_{17}=3$&${\mathfrak n}_{18}=4$&${\mathfrak n}_{19}=3$\\
\cline{2-11}
&${\mathfrak n}_{20}=4$&${\mathfrak n}_{21}=2$&${\mathfrak n}_{22}=3$&${\mathfrak n}_{23}=1$&${\mathfrak n}_{24}=3$&${\mathfrak n}_{25}=0$&${\mathfrak n}_{26}=2$&${\mathfrak n}_{27}=0$&${\mathfrak n}_{28}=1$&${\mathfrak n}_{29}=0$\\
\cline{2-11}
&${\mathfrak c}_1=14$&${\mathfrak c}_2=21$&${\mathfrak c}_3=42$&${\mathfrak c}_4=25$&\multicolumn{6}{|c|}{{\small${\mathfrak s}=(1336,282,102,28,4)$}}\\
\hline
\multirow{4}{*}{$G_2$}&${\mathfrak n}_0=1$&${\mathfrak n}_1=1$&${\mathfrak n}_2=3$&${\mathfrak n}_3=2$&${\mathfrak n}_4=4$&${\mathfrak n}_5=2$&${\mathfrak n}_6=4$&${\mathfrak n}_7=2$&${\mathfrak n}_8=4$&${\mathfrak n}_9=2$\\
\cline{2-11}
&${\mathfrak n}_{10}=4$&${\mathfrak n}_{11}=2$&${\mathfrak n}_{12}=4$&${\mathfrak n}_{13}=2$&${\mathfrak n}_{14}=4$&${\mathfrak n}_{15}=2$&${\mathfrak n}_{16}=4$&${\mathfrak n}_{17}=2$&${\mathfrak n}_{18}=4$&${\mathfrak n}_{19}=2$\\
\cline{2-11}
&${\mathfrak n}_{20}=3$&${\mathfrak n}_{21}=1$&${\mathfrak n}_{22}=2$&${\mathfrak n}_{23}=1$&${\mathfrak n}_{24}=2$&${\mathfrak n}_{25}=0$&${\mathfrak n}_{26}=1$&${\mathfrak n}_{27}=0$&${\mathfrak n}_{28}=1$&${\mathfrak n}_{29}=0$\\
\cline{2-11}
&${\mathfrak c}_1=14$&${\mathfrak c}_2=21$&${\mathfrak c}_3=24$&${\mathfrak c}_4=7$&\multicolumn{6}{|c|}{{\small${\mathfrak s}=(873,156,66,28,4)$}}\\
\hline
\end{tabular}}
\end{center}
\end{stepsb}

\vspace{2mm}
\begin{steps}{The ``new'' map.} 
In the previous section we have proposed a new polynomial map $P=(P_1,P_2):\R^2\to\R^2$, obtained by composing six suitable maps, and we have proved in a tricky way, 
but without computational effort, that its image is the open quadrant. As it can be expected, when composing six maps, even if they are quite simple, the size of their 
composition easily becomes very large. Namely, let us take a look at the coordinates of $P$ and we prevent the reader that the result is rather amazing:
{\tiny\begin{align*} 
P_1(\x,\y)=&\y^{14}\x^{10}+\y^{13}\big(-10\x^{9}\big)+\y^{12}\big(3\x^{10}+45\x^{8}\big)+\y^{11}\big(-24\x^{9}-120\x^{7}\big)+\y^{10}\big(3\x^{10}+84\x^{8}+210\x^{6}\big)\\
&+\y^{9}\big(-18\x^{9}-168\x^{7}-252\x^{5}\big)
+\y^{8}\big(\x^{10}+45\x^{8}+208\x^{6}+210\x^{4}\big)+\y^{7}\big(-4\x^{9}-60\x^{7}-156\x^{5}-120\x^{3}\big)\\
&
+\y^{6}\big(6\x^{8}+41\x^{6}+54\x^{4}+45\x^{2}\big)
+\y^{5}\big(-4\x^{7}-2\x^{5}+16\x^{3}-10\x\big)+\y^{4}\big(-\x^{6}-21\x^{4}-26\x^{2}+1\big)\\
&+\y^{3}\big(4\x^{5}+16\x^{3}+10\x\big)+\y^{2}\big(-2\x^{4}-3\x^{2}-1\big)+\y\big(-2\x\big)+\big(\x^{2}+1\big)\\[4pt]
P_2(\x,\y)=&3\y^{58}\x^{42}
+\y^{57}\big(-126\x^{41}\big)
+\y^{56}\big(39\x^{42}+2583\x^{40}\big)
+\y^{55}\big(-1560\x^{41}-34440\x^{39}\big)\\
+\y^{54}&\big(234\x^{42}+30420\x^{40}+335790\x^{38}\big)
+\y^{53}\big(-8892\x^{41}-385320\x^{39}-2552004\x^{37}\big)\\
+\y^{52}&\big(858\x^{42}+164502\x^{40}+3564186\x^{38}+15737358\x^{36}\big)
+\y^{51}\big(-30888\x^{41}-1974024\x^{39}-25661400\x^{37}-80934984\x^{35}\big)
\\
+\y^{50}&\big(2145\x^{42}+540540\x^{40}+17272422\x^{38}+149679948\x^{36}+354090555\x^{34}\big)
\\
+\y^{49}&\big(-72930\x^{41}-6126120\x^{39}-117444060\x^{37}-726896376\x^{35}-1337675430\x^{33}\big)
\\
+\y^{48}&\big(3861\x^{42}+1203345\x^{40}+50538906\x^{38}+645817914\x^{36}+2997511167\x^{34}+4414328919\x^{32}\big)
\\
+\y^{47}&\big(-123552\x^{41}-12835680\x^{39}-323405280\x^{37}-2951083584\x^{35}-10652070120\x^{33}-12841684128\x^{31}\big)
\\
+\y^{46}&\big(5148\x^{42}+1915056\x^{40}+99471240\x^{38}+1670316912\x^{36}+11426452689\x^{34}+32992510980\x^{32}+33174350664\x^{30}\big)
\\
+\y^{45}&\big(-154440\x^{41}-19150560\x^{39}-596690160\x^{37}-7152830784\x^{35}-38036156658\x^{33}-89857441824\x^{31}\\
&-76556193840\x^{29}\big)
\\
+\y^{44}&\big(5148\x^{42}+2239380\x^{40}+138829680\x^{38}+2882200200\x^{36}+25889913567\x^{34}+110058839769\x^{32}+216714158424\x^{30}\\
&+158580687240\x^{28}\big)
\\
+\y^{43}&\big(-144144\x^{41}-20900880\x^{39}-777156336\x^{37}-11513087520\x^{35}-80334221448\x^{33}-279174363456\x^{31}\\
&-465386752032\x^{29}-296017282848\x^{27}\big)
\\
+\y^{42}&\big(3861\x^{42}+1945944\x^{40}+141061932\x^{38}+3493639512\x^{36}+38755194675\x^{34}+215962671348\x^{32}+624842278296\x^{30}\\
&+893740792464\x^{28}+499529164806\x^{26}\big)
\\
+\y^{41}&\big(-100386\x^{41}-16864848\x^{39}-733088664\x^{37}-12947337216\x^{35}-111444852810\x^{33}-506976243984\x^{31}\\
&-1240031806320\x^{29}-1539973277664\x^{27}-763985781468\x^{25}\big)
\\
+\y^{40}&\big(2145\x^{42}+1254825\x^{40}+105383124\x^{38}+3049568676\x^{36}+40285600785\x^{34}+276487648785\x^{32}+1045226877726\x^{30}\\
&+2189687493816\x^{28}+2386470435246\x^{26}+1061091363150\x^{24}\big)
\\
+\y^{39}&\big(-51480\x^{41}-10038600\x^{39}-505368864\x^{37}-10418028192\x^{35}-106603885260\x^{33}-595930780260\x^{31}\\
&-1899831043908\x^{29}-3448088489088\x^{27}-3331233381168\x^{25}-1340325932400\x^{23}\big)
\\
+\y^{38}&\big(858\x^{42}+592020\x^{40}+57702942\x^{38}+1932250320\x^{36}+29741698854\x^{34}+242033582610\x^{32}+1120834360983\x^{30}\\
&+3050643769218\x^{28}+4846401714402\x^{26}+4191334449480\x^{24}+1541374822260\x^{22}\big)
\\
+\y^{37}&\big(-18876\x^{41}-4341480\x^{39}-253520388\x^{37}-6037780320\x^{35}-71785869516\x^{33}-474028549020\x^{31}\\
&-1843203779238\x^{29}-4328177122488\x^{27}-6076884293892\x^{25}-4752136223280\x^{23}-1614773623320\x^{21}\big)
\\
+\y^{36}&\big(234\x^{42}+198198\x^{40}+22780890\x^{38}+883671822\x^{36}+15669033282\x^{34}+147520185174\x^{32}+802564715817\x^{30}\\
&+2648349484425\x^{28}+5414076477567\x^{26}+6782238651330\x^{24}+4848835486140\x^{22}+1541374822260\x^{20}\big)
\\
+\y^{35}&\big(-4680\x^{41}-1321320\x^{39}-90909720\x^{37}-2501293608\x^{35}-34098937488\x^{33}-258790726656\x^{31}\\
&-1172719484748\x^{29}-3310563929400\x^{27}-5939358094890\x^{25}-6705746117280\x^{23}-4440402291600\x^{21}\\
&-1340325932400\x^{19}\big)
\\
+\y^{34}&\big(39\x^{42}+44460\x^{40}+6270990\x^{38}+285964140\x^{36}+5830003629\x^{34}+62463100392\x^{32}+386619981912\x^{30}\\
&+1468889520102\x^{28}+3565792063237\x^{26}+5654948284161\x^{24}+5823675651600\x^{22}+3632661903240\x^{20}\\
&+1061091363150\x^{18}\big)
\\
\end{align*}
\begin{align*}
+\y^{33}&\big(-702\x^{41}-266760\x^{39}-22488972\x^{37}-724097880\x^{35}-11256823602\x^{33}-96041328768\x^{31}-487123552512\x^{29}\\
&-1552735657368\x^{27}-3244819146854\x^{25}-4580187156504\x^{23}-4374531115920\x^{21}-2634548052240\x^{19}\\
&-763985781468\x^{17}\big)
\\
+\y^{32}&\big(3\x^{42}+5967\x^{40}+1132146\x^{38}+63014754\x^{36}+1490710365\x^{34}+17959126149\x^{32}+122382223740\x^{30}\\
&+505745858592\x^{28}+1338543457479\x^{26}+2389903772251\x^{24}+3024079031664\x^{22}+2757678252120\x^{20}\\
&+1671807486870\x^{18}+499529164806\x^{16}\big)
\\
+\y^{31}&\big(-48\x^{41}-31824\x^{39}-3599424\x^{37}-140307552\x^{35}-2492292480\x^{33}-23311230408\x^{31}-125118290808\x^{29}\\
&-408942989136\x^{27}-860992063020\x^{25}-1264844514452\x^{23}-1448403199200\x^{21}-1358500804560\x^{19}\\
&-906044290704\x^{17}-296017282848\x^{15}\big)
\\
+\y^{30}&\big(360\x^{40}+119052\x^{38}+8827488\x^{36}+248794545\x^{34}+3325653000\x^{32}+23589221814\x^{30}+93943097532\x^{28}\\
&+214105273416\x^{26}+279000958914\x^{24}+232526141385\x^{22}+251771240844\x^{20}+402459623100\x^{18}\\
&+397758414600\x^{16}+158580687240\x^{14}\big)
\\
+\y^{29}&\big(-1680\x^{39}-329544\x^{37}-16847136\x^{35}-345167730\x^{33}-3359079000\x^{31}-16310612412\x^{29}-34739738208\x^{27}\\
&+8707395600\x^{25}+201636600516\x^{23}+427424131598\x^{21}+391326052176\x^{19}+88115205096\x^{17}\\
&-120256425120\x^{15}-76556193840\x^{13}\big)
\\
+\y^{28}&\big(5436\x^{38}+689436\x^{36}+24634407\x^{34}+352228713\x^{32}+2132672178\x^{30}+2848700394\x^{28}-29503694346\x^{26}\\
&-172452016656\x^{24}-439850599793\x^{22}-640536891329\x^{20}-549187064457\x^{18}-235066082508\x^{16}\\
&+2521695000\x^{14}+33174350664\x^{12}\big)
\\
+\y^{27}&\big(-12768\x^{37}-1079856\x^{35}-25802376\x^{33}-206928096\x^{31}+136680552\x^{29}+11170302192\x^{27}+70881386772\x^{25}\\
&+224167054032\x^{23}+426758710668\x^{21}+527078791136\x^{19}+427235418810\x^{17}+202941178848\x^{15}\\
&+28662745824\x^{13}-12841684128\x^{11}\big)
\\
+\y^{26}&\big(21840\x^{36}+1183293\x^{34}+14363220\x^{32}-69202680\x^{30}-2434311540\x^{28}-18945775935\x^{26}-74285129058\x^{24}\\
&-173304693008\x^{22}-265095578422\x^{20}-288533573070\x^{18}-230403919221\x^{16}-120314905056\x^{14}\\
&-25340394480\x^{12}+4414328919\x^{10}\big)
\\
+\y^{25}&\big(-25584\x^{35}-650670\x^{33}+8607408\x^{31}+345304032\x^{29}+3513797040\x^{27}+17163842958\x^{25}+46639735632\x^{23}\\
&+76708172944\x^{21}+86659910704\x^{19}+84926445640\x^{17}+79302027552\x^{15}+52673077728\x^{13}+14782445472\x^{11}\\
&-1337675430\x^{9}\big)
\\
+\y^{24}&\big(14667\x^{34}-518661\x^{32}-31722402\x^{30}-454299168\x^{28}-2814843483\x^{26}-8355817359\x^{24}-10078728144\x^{22}\\
&+4426737232\x^{20}+24802938164\x^{18}+21115198600\x^{16}-3895368354\x^{14}-15657924960\x^{12}\\
&-6772789749\x^{10}+354090555\x^{8}\big)
\\
+\y^{23}&\big(12756\x^{33}+1729260\x^{31}+39486300\x^{29}+324117360\x^{27}+931942872\x^{25}-1227016800\x^{23}-14597369680\x^{21}\\
&-40114186080\x^{19}-57009589860\x^{17}-45459836348\x^{15}-16944251796\x^{13}+1414444896\x^{11}+2555765448\x^{9}\\
&-80934984\x^{7}\big)
\\
+\y^{22}&\big(-42702\x^{32}-2092173\x^{30}-25150158\x^{28}-48126912\x^{26}+849422100\x^{24}+6344498132\x^{22}+20213240376\x^{20}\\
&+36768700761\x^{18}+42303355110\x^{16}+31703950597\x^{14}+13927253922\x^{12}+1746966924\x^{10}-806112540\x^{8}\\
&+15737358\x^{6}\big)
\\
+\y^{21}&\big(51444\x^{31}+1194786\x^{29}-1752264\x^{27}-184758408\x^{25}-1553215392\x^{23}-5979456960\x^{21}-12992319872\x^{19}\\
&-17906216874\x^{17}-17554434852\x^{15}-13118660562\x^{13}-6658703640\x^{11}-1391904144\x^{9}+212668440\x^{7}-2552004\x^{5}\big)
\\
+\y^{20}&\big(-26667\x^{30}+389709\x^{28}+22313997\x^{26}+242679348\x^{24}+1144713740\x^{22}+2737085536\x^{20}+3460510619\x^{18}\\
&+2481721485\x^{16}+1863331741\x^{14}+2400699731\x^{12}+2046794679\x^{10}+642414528\x^{8}-46553286\x^{6}+335790\x^{4}\big)
\\
+\y^{19}&\big(-16284\x^{29}-1501752\x^{27}-24187170\x^{25}-142544760\x^{23}-307848448\x^{21}+165305400\x^{19}+1926217032\x^{17}\\
&+3590019624\x^{15}+3043120640\x^{13}+976243768\x^{11}-270645486\x^{9}-218577144\x^{7}+8322216\x^{5}-34440\x^{3}\big)
\\
+\y^{18}&\big(44430\x^{28}+1416474\x^{26}+10950405\x^{24}+3964374\x^{22}-266834728\x^{20}-1255149323\x^{18}-2711797820\x^{16}\\
&-3343104394\x^{14}-2491865192\x^{12}-1032987313\x^{10}-111159765\x^{8}+57844626\x^{6}-1184628\x^{4}+2583\x^{2}\big)
\\
\end{align*}
\begin{align*}
+\y^{17}&\big(-37464\x^{27}-450144\x^{25}+3711840\x^{23}+69728740\x^{21}+365201224\x^{19}+948027650\x^{17}+1432282856\x^{15}\\
&+1408325204\x^{13}+980400160\x^{11}+456664614\x^{9}+90942768\x^{7}-12010116\x^{5}+129096\x^{3}-126\x\big)
\\
+\y^{16}&\big(6580\x^{26}-461400\x^{24}-9448118\x^{22}-62221518\x^{20}-188957177\x^{18}-293799569\x^{16}-250061840\x^{14}-160510774\x^{12}\\
&-144512388\x^{10}-112388243\x^{8}-35387718\x^{6}+1930710\x^{4}-10101\x^{2}+3\big)
\\
+\y^{15}&\big(18988\x^{25}+689004\x^{23}+6319728\x^{21}+20662928\x^{19}+11193988\x^{17}-78909172\x^{15}-190998084\x^{13}-181793728\x^{11}\\
&-70141034\x^{9}+5124268\x^{7}+9406872\x^{5}-232656\x^{3}+504\x\big)
\\
+\y^{14}&\big(-21530\x^{24}-348863\x^{22}-817348\x^{20}+8381100\x^{18}+53465322\x^{16}+130750043\x^{14}+169818186\x^{12}+128080288\x^{10}\\
&+54450093\x^{8}+8357617\x^{6}-1813536\x^{4}+19803\x^{2}-12\big)
\\
+\y^{13}&\big(7732\x^{23}-51794\x^{21}-1979632\x^{19}-13075536\x^{17}-38242716\x^{15}-60150326\x^{13}-57313624\x^{11}-37472436\x^{9}\\
&-17369748\x^{7}-4022762\x^{5}+252096\x^{3}-1062\x\big)
\\
+\y^{12}&\big(4839\x^{22}+190543\x^{20}+1684709\x^{18}+5904504\x^{16}+9212111\x^{14}+5589481\x^{12}+161348\x^{10}+519282\x^{8}+2168175\x^{6}\\
&+1063995\x^{4}-24102\x^{2}+27\big)
\\
+\y^{11}&\big(-7124\x^{21}-111104\x^{19}-428818\x^{17}+106776\x^{15}+3774512\x^{13}+8627280\x^{11}+8666592\x^{9}+4118664\x^{7}+567980\x^{5}\\
&-185128\x^{3}+1422\x\big)
\\
+\y^{10}&\big(2818\x^{20}+3325\x^{18}-248311\x^{16}-1543218\x^{14}-3882088\x^{12}-5089379\x^{10}-3823968\x^{8}-1696258\x^{6}-358242\x^{4}\\
&+21256\x^{2}-39\big)
\\
+\y^{9}&\big(912\x^{19}+33346\x^{17}+251336\x^{15}+770668\x^{13}+1164024\x^{11}+959578\x^{9}+529724\x^{7}+272716\x^{5}+91400\x^{3}-1480\x\big)
\\
+\y^{8}&\big(-1469\x^{18}-18869\x^{16}-67896\x^{14}-59802\x^{12}+118483\x^{10}+280207\x^{8}+193737\x^{6}+29334\x^{4}-13689\x^{2}+48\big)
\\
+\y^{7}&\big(492\x^{17}+868\x^{15}-21252\x^{13}-107544\x^{11}-214620\x^{9}-217276\x^{7}-113886\x^{5}-25312\x^{3}+1178\x\big)
\\
+\y^{6}&\big(158\x^{16}+3769\x^{14}+21194\x^{12}+51413\x^{10}+65430\x^{8}+47519\x^{6}+21585\x^{4}+5925\x^{2}-45\big)
\\
+\y^{5}&\big(-180\x^{15}-1650\x^{13}-4576\x^{11}-4426\x^{9}+868\x^{7}+3850\x^{5}+896\x^{3}-694\x\big)
\\
+\y^{4}&\big(35\x^{14}-45\x^{12}-1310\x^{10}-4311\x^{8}-6493\x^{6}-4755\x^{4}-1308\x^{2}+35\big)
\\
+\y^{3}&\big(20\x^{13}+240\x^{11}+898\x^{9}+1704\x^{7}+1832\x^{5}+1032\x^{3}+258\x\big)
+\y^{2}\big(-10\x^{12}-55\x^{10}-109\x^{8}-114\x^{6}-40\x^{4}-3\x^{2}-17\big)
\\
+\y&\big(-10\x^{9}-40\x^{7}-84\x^{5}-88\x^{3}-42\x\big)
+\big(\x^{10}+5\x^{8}+14\x^{6}+22\x^{4}+21\x^{2}+9\big)
\end{align*}}
Let us compute the sizes of $P_1$ and $P_2$ and the involved numbers ${\mathfrak n}_i,{\mathfrak c}_j$ (see \ref{size}). 
\begin{center}
{\SMALL
\renewcommand*{\arraystretch}{1.5}
\begin{tabular}{|c|c|c|c|c|c|c|c|c|c|c|c|c|}
\hline
\multirow{4}{*}{$P_1$}&${\mathfrak n}_0=1$&${\mathfrak n}_1=0$&${\mathfrak n}_2=3$&${\mathfrak n}_3=0$&${\mathfrak n}_4=3$&${\mathfrak n}_5=0$&${\mathfrak n}_6=4$&${\mathfrak n}_7=0$&${\mathfrak n}_8=4$&${\mathfrak n}_9=0$\\
\cline{2-11}
&${\mathfrak n}_{10}=4$&${\mathfrak n}_{11}=0$&${\mathfrak n}_{12}=4$&${\mathfrak n}_{13}=0$&${\mathfrak n}_{14}=4$&${\mathfrak n}_{15}=0$&${\mathfrak n}_{16}=4$&${\mathfrak n}_{17}=4$&${\mathfrak n}_{18}=0$&${\mathfrak n}_{19}=4$\\
\cline{2-11}
&${\mathfrak n}_{20}=3$&${\mathfrak n}_{21}=0$&${\mathfrak n}_{22}=2$&${\mathfrak n}_{23}=0$&${\mathfrak n}_{24}=1$&${\mathfrak n}_{25}=0$&${\mathfrak n}_{26}=0$&${\mathfrak n}_{27}=0$&${\mathfrak n}_{28}=0$&${\mathfrak n}_{29}=0$\\
\cline{2-11}
&${\mathfrak c}_1=17$&${\mathfrak c}_2=16$&${\mathfrak c}_3=8$&\multicolumn{7}{|c|}{{\small${\mathfrak s}=(523,73,41,24,3)$}}\\
\hline
\multirow{4}{*}{$P_2$}&${\mathfrak n}_0=1$&${\mathfrak n}_1=0$&${\mathfrak n}_2=3$&${\mathfrak n}_3=0$&${\mathfrak n}_4=5$&${\mathfrak n}_5=0$&${\mathfrak n}_6=7$&${\mathfrak n}_7=0$&${\mathfrak n}_8=9$&${\mathfrak n}_9=0$\\
\cline{2-11}
&${\mathfrak n}_{10}=11$&${\mathfrak n}_{11}=0$&${\mathfrak n}_{12}=11$&${\mathfrak n}_{13}=0$&${\mathfrak n}_{14}=13$&${\mathfrak n}_{15}=0$&${\mathfrak n}_{16}=14$&${\mathfrak n}_{17}=0$&${\mathfrak n}_{18}=14$&${\mathfrak n}_{19}=0$\\
\cline{2-11}
&${\mathfrak n}_{20}=14$&${\mathfrak n}_{21}=0$&${\mathfrak n}_{22}=14$&${\mathfrak n}_{23}=0$&${\mathfrak n}_{24}=14$&${\mathfrak n}_{25}=0$&${\mathfrak n}_{26}=14$&${\mathfrak n}_{27}=0$&${\mathfrak n}_{28}=14$&${\mathfrak n}_{29}=0$\\
\cline{2-11}
&${\mathfrak n}_{30}=14$&${\mathfrak n}_{31}=0$&${\mathfrak n}_{32}=14$&${\mathfrak n}_{33}=0$&${\mathfrak n}_{34}=14$&${\mathfrak n}_{35}=0$&${\mathfrak n}_{36}=14$&${\mathfrak n}_{37}=0$&${\mathfrak n}_{38}=14$&${\mathfrak n}_{39}=0$\\
\cline{2-11}
&${\mathfrak n}_{40}=14$&${\mathfrak n}_{41}=0$&${\mathfrak n}_{42}=14$&${\mathfrak n}_{43}=0$&${\mathfrak n}_{44}=14$&${\mathfrak n}_{45}=0$&${\mathfrak n}_{46}=14$&${\mathfrak n}_{47}=0$&${\mathfrak n}_{48}=14$&${\mathfrak n}_{49}=0$\\
\cline{2-11}
&${\mathfrak n}_{50}=14$&${\mathfrak n}_{51}=0$&${\mathfrak n}_{52}=14$&${\mathfrak n}_{53}=0$&${\mathfrak n}_{54}=14$&${\mathfrak n}_{55}=0$&${\mathfrak n}_{56}=14$&${\mathfrak n}_{57}=0$&${\mathfrak n}_{58}=14$&${\mathfrak n}_{59}=0$\\
\cline{2-11}
&${\mathfrak n}_{60}=14$&${\mathfrak n}_{61}=0$&${\mathfrak n}_{62}=14$&${\mathfrak n}_{63}=0$&${\mathfrak n}_{64}=14$&${\mathfrak n}_{65}=0$&${\mathfrak n}_{66}=14$&${\mathfrak n}_{67}=0$&${\mathfrak n}_{68}=14$&${\mathfrak n}_{69}=0$\\
\cline{2-11}
&${\mathfrak n}_{70}=14$&${\mathfrak n}_{71}=0$&${\mathfrak n}_{72}=14$&${\mathfrak n}_{73}=0$&${\mathfrak n}_{74}=14$&${\mathfrak n}_{75}=0$&${\mathfrak n}_{76}=13$&${\mathfrak n}_{77}=0$&${\mathfrak n}_{78}=12$&${\mathfrak n}_{79}=0$\\
\cline{2-11}
&${\mathfrak n}_{80}=11$&${\mathfrak n}_{81}=0$&${\mathfrak n}_{82}=10$&${\mathfrak n}_{83}=0$&${\mathfrak n}_{84}=9$&${\mathfrak n}_{85}=0$&${\mathfrak n}_{86}=8$&${\mathfrak n}_{87}=0$&${\mathfrak n}_{88}=7$&${\mathfrak n}_{89}=0$\\
\cline{2-11}
&${\mathfrak n}_{90}=6$&${\mathfrak n}_{91}=0$&${\mathfrak n}_{92}=5$&${\mathfrak n}_{93}=0$&${\mathfrak n}_{94}=4$&${\mathfrak n}_{95}=0$&${\mathfrak n}_{96}=3$&${\mathfrak n}_{97}=0$&${\mathfrak n}_{98}=2$&${\mathfrak n}_{99}=0$\\
\cline{2-11}
&${\mathfrak n}_{100}=1$&${\mathfrak c}_{1}=7$&${\mathfrak c}_{2}=24$&${\mathfrak c}_{3}=22$&${\mathfrak c}_{4}=39$&${\mathfrak c}_{5}=41$&${\mathfrak c}_{6}=48$&${\mathfrak c}_{7}=55$&${\mathfrak c}_{8}=49$&${\mathfrak c}_{9}=57$\\
\cline{2-11}
&${\mathfrak c}_{10}=56$&${\mathfrak c}_{11}=65$&${\mathfrak c}_{12}=58$&${\mathfrak c}_{13}=50$&\multicolumn{6}{|c|}{{\small${\mathfrak s}=(27679,4681,571,100,13)$}}\\
\hline
\end{tabular}}
\end{center}
\end{steps}

Thus, as one could expect, the size of the ``new'' map $P$ is much larger than the size of the``old'' map $G=F\circ F'$. Hence, for its possible computational applications, the ``old'' map is rather better than the ``new one'' $P$. Even more, since $F(\R^2)=\Qq\cup\{(1,0),(0,1)\}$ and $F^{-1}\big(\{(1,0),(0,1)\}\big)=\{(-1,0),(0,-1)\}$ (see \ref{old}), for all computational applications one can use $F$, whose size is smaller, instead of $G$, with the caution that $F^{-1}(\{(1,0),(0,1)\})=\{(-1,0),(0,-1)\}$. However, it must be said that, in our opinion, the ``new'' map has the great advantage, from a theoretical point of view, that the checking of its surjectivity does not involve the computation of Sturm sequences with polynomials of high degree. As it is well-known, these require algorithms of high computational complexity and may induce some readers to think that the proof of \ref{mainthm} provided in \cite{fg1} has an empirical flavour. 

\section{Complement of an open ball}\label{s4}

It follows from \cite[3.4]{fg2} that the complement of a closed ball in $\R^n$ is not a polynomial image of $\R^n$. Indeed, consider the $n$-dimensional closed ball $\ol{\ball}_n$ of radius $1$ centered at the origin of $\R^n$, and its complement $S=\R^n\setminus\ol{\ball}$. Choose the $(n-1)$-dimensional algebraic set
$$
X=\partial S=\cl_{\R^n}(S)\setminus S=\{(x_1,\ldots,x_n)\in\R^n:\ x_1^2+\cdots+x_n^2=1\}.
$$
Since $X\cap\cl_{\R^n}(S)=X$ is a bounded set and $\dim(X\cap\partial S)=n-1$, it follows from \cite[3.4]{fg2} that $S$ is not a polynomial image of $\R^n$.

On the contrary, we will prove in this section, using \ref{mainthm}, that the complement of an open ball in $\R^n$ is a polynomial image of $\R^n$. In the bidimensional case this was proved in \cite[4.1]{fg2}, but it seems rather difficult to make suitable its proof for the $n$-dimensional case. Later in \cite [6.1]{fgu1} we present a different strategy to prove that the complement of an open ball in $\R^3$ is a polynomial image of $\R^3$, and this is the one we are going to generalize to the $n$-dimensional setting for $n\geq2$. Before proving that we need some preliminary results.

It has been already proved in \cite[1.1]{u2} that the complement of the square $\cube_2=\{(x,y)\in\R^2:\ |x|\le1,\ |y|\le1\}\subset\R^2$ is a polynomial image of $\R^2$. However, we show here a proof that illustrates the use of \ref{mainthm}, together with ideas further developed in \cite{u2}.

\begin{lem}\label{n2}
The complement of the square $\cube_2=\{(x,y)\in\R^2: |x|\le1,\ |y|\le1\}$ is a polynomial image of $\R^2$.
\end{lem}
\begin{proof}
First, we know from \ref{mainthm} that the open quadrant $\Qq=\{(x_1,y_1)\in\R^2:\ x_1>0,\ y_1>0\}$ is the image of a polynomial map $F_1:\R^2\to\R^2$. Next, we identify $\R^2$ with $\C$ and consider the polynomial map 
$$
\begin{array}{cccl}
F_2:&\R^2\equiv\C\hspace{9.75mm}&\to&\hspace{22.1mm}\C\equiv\R^2,\\ 
&(x_1,y_1)\equiv x_1+\sqrt{-1}y_1&\mapsto&(x_1+\sqrt{-1}y_1)^3\equiv(x_1^3-3x_1y_1^2,3x_1^2y_1-y_1^3)=(x_2,y_2).
\end{array}
$$
A straightforward computation shows that $F_2(\Qq)=A=\R^2\setminus\{(x_2,y_2)\in\R^2:\ x_2\geq0, y_2\leq0\}$. Now, notice that the affine bijection
$$
F_3:\R^2\to\R^2,\ (x_2,y_2)\to(x_3,y_3)=(x_2-y_2-1,x_2+y_2)
$$
maps $A$ onto 
$$
B=\R^2\setminus\{(x_3,y_3)\in\R^2:\ x_3+y_3+1\geq0,\ x_3-y_3+1\geq0\}.
$$ Next, consider the polynomial map 
$$
F_4:\R^2\to\R^2,\ (x_3,y_3)\to(x_4,y_4)=\big(x_3,y_3(1-x_3(x_3+y_3+1)^2(x_3-y_3+1)^2)\big)
$$
and observe that, by \cite[2.5]{u2}, $F_4(B)=\R^2\setminus\trian$ where 
$$
\trian=\{(x_4,y_4)\in\R^2:\ x_4+y_4+1\geq0,\ x_4-y_4+1\geq0,\ x_4\leq0\}.
$$ 
Now, we need to place properly this triangle, and to that end we consider the affine bijection
$$
F_5:\R^2\to\R^2,\ (x_4,y_4)\mapsto(x_5,y_5)=(x_4-y_4,x_4+y_4+1),
$$
which maps $\trian$ onto $\trian'=\{(x_5,y_5)\in\R^2:\ x_5+1\geq0,\ y_5\geq0,\ -x_5-y_5+1\geq0\}$, and so $F_5(\R^2\setminus\trian)=\R^2\setminus\trian'$. 

Next, again by \cite[2.5]{u2}, the image of $\R^2\setminus\trian'$ under the polynomial ~map
$$
F_6:\R^2\to\R^2,\ (x_5,y_5)\mapsto(x_6,y_6)=\big(x_5,y_5(1-x_5y_5^2(x_5+y_5-1)^2)\big)
$$
is the complement of the quadrilateral 
$$
\pol=\{(x_6,y_6)\in\R^2:\ x_6+1\geq0,\ y_6\geq0,\ -x_6-y_6+1\geq0,\ x_6\leq0\}.
$$
Again, we need to place properly the quadrilateral $\pol$ to proceed and we consider the affine bijection
$$
F_7:\R^2\to\R^2,\ (x_6,y_6)\mapsto(x_7,y_7)=(y_6-1,x_6+1)
$$
which maps $\pol$ onto 
$$
\pol'=\{(x_7,y_7)\in\R^2:\ x_7+1\geq0,\ 0\leq y_7\leq1,\ -x_7-y_7+1\geq0\},
$$ 
and consequently $F_7(\R^2\setminus \pol)=\R^2\setminus\pol'$. Now, using \cite[2.5]{u2} once more, it follows that the image of $\R^2\setminus \pol'$ under the polynomial map
$$
F_8:\R^2\to\R^2,\ (x_7,y_7)\mapsto(x_8,y_8)=(x_7,y_7\big(1-x_7y_7^2(y_7-1)^2(x_7+y_7-1)^2)\big)
$$
is the complement of the square 
$$
\squarea=[-1,0]\times[0,1]=\{(x_8,y_8)\in\R^2:\ -1\leq x_8\leq0,\ 0\leq y_8\leq1\}.
$$ 
Next, the affine bijection
$$
F_9:\R^2\to\R^2,\ (x_8,y_8)\mapsto(x_9,y_9)=(2x_8+1,2y_8-1),
$$
maps $\squarea$ onto $\cube_2$ and so $F_9(\R^2\setminus\squarea)=\R^2\setminus\cube_2$.

Once all this is done, it is straightforward to check that the image of the polynomial map $F=F_9\circ\cdots\circ F_1:\R^2\to\R^2$ is $\R^2\setminus\cube_2$, and we are done.
\end{proof}

\begin{prop}\label{box} 
The complement of the $n$-dimensional cube $\cube_n=\{(x_1,\dots, x_n)\in\R^n:\ |x_i|\le1,\ 1\le i\le n\}$ is a polynomial image of $\R^n$ for $n\geq2$.
\end{prop}
\begin{proof}
We proceed by induction on $n$. For $n=2$ the result has been just proved in \ref{n2}. Let us assume that it is true for $k=n-1\geq2$ and let us check that it also holds for $n$. Let $f_1:\R^{n-1}\rightarrow\R^{n-1}$ be a polynomial map such that $f_1(\R^{n-1})=\R^{n-1}\setminus\cube_{n-1}$. Next, consider the polynomial map 
$$
F_1:\R^n\rightarrow\R^n,\ x=(x_1,\dots,x_n)\mapsto\big(f_1(x_1,\dots,x_{n-1}), x_n\big),
$$
whose image is $\R^n\setminus(\cube_{n-1}\times\R)$. Next, we choose the polynomial map 
$$
F_2:\R^n\rightarrow\R^n,\ x=(x_1,\dots,x_n)\mapsto\big(g(x)x_1,\dots,g(x)x_{n-1},x_n\big),
$$
where $g(x)=1+(1-x_n^2)\prod_{i=1}^{n-1}(x_i^2-1)^2$. We claim that: $F_2\big(\R^n\setminus(\cube_{n-1}\times\R)\big)=\R^n\setminus\cube_n$. Once this is proved we deduce that the image of the polynomial map $F_2\circ F_1:\R^n\to\R^n$ is $\R^n\setminus\cube_n$.
Thus, it only remains to prove the claim.

Indeed, observe first that if we set $H_k=\{x\in\R^n:\ x_n=k\}$, then $F_2(H_k)\subset H_k$. Moreover, if $\Delta=\{x\in\R^n:\ g(x)=1\}$ we have the equality $\Delta=\bigcup_{i=1}^n\{x\in\R^n:\ x_i^2=1\}$. Thus, since $\partial\cube_n=\cube_n\cap\Delta$, it follows that $g|_{\partial\cube_n}\equiv1$ and so $F_2|_{\partial\cube_n}=\Id|_{\partial\cube_n}$. Furthermore, $g(x)>1$ if and only if $|x_n|<1$. 

Next, for each point $y=(x_1,\ldots,x_{n-1})\in\partial \cube_{n-1}$ and each $k\in\R$, we define the half-lines $S_{y,k}=\{(ty,k)\in\R^n:\ t>1\}$ and $T_{y,k}=\{(ty,k)\in\R^n:\ t<1\}$ which are contained in the line $L_{y,k}=\{(ty,k)\in\R^n:\ t\in\R\}\subset H_k$. We denote $A_n=\{x\in\R^n:\ x_n^2>1\}$. Observe that 

\begin{align*}
\R^n\setminus(\cube_{n-1}\times\R)=(\R^{n-1}\setminus\cube_{n-1})\times\R&=\bigcup_{k\in\R}(\R^{n-1}\setminus\cube_{n-1})\times\{k\},\\[5pt]
\R^n\setminus\cube_n=\big(\R^n\setminus(\cube_{n-1}\times\R)\big)\cup A_n&=\big((\R^{n-1}\setminus\cube_{n-1})\times\R\big)\cup A_n\\
&=\bigcup_{|k|\leq1}(\R^{n-1}\setminus\cube_{n-1})\times\{k\}\cup\bigcup_{|k|>1}H_k.
\end{align*}

Moreover, for any fixed $k\in\R$, a straightforward computation shows that for each point $y\in\partial\cube_{n-1}$,
$$
(\R^{n-1}\setminus\cube_{n-1})\times\{k\}=\bigcup_{y\in\partial\cube_{n-1}}S_{y,k}\quad\text{and}\quad H_k=\bigcup_{y\in\partial\cube_{n-1}}E_{y,k}
$$ 
for every set $E_{y,k}$ satisfying $T_{y,k}\subset E_{y,k}\subset L_{y,k}$. Thus, the equalities 
\begin{align*}
&\R^n\setminus(\cube_{n-1}\times\R)=\bigcup_{k\in\R}\ \bigcup_{y\in\partial\cube_{n-1}}S_{y,k},\\
&\R^n\setminus\cube_n=\Big(\bigcup_{|k|\leq1}\ \bigcup_{y\in\partial\cube_{n-1}}S_{y,k}\Big)\cup\Big(\bigcup_{|k|>1}\ \bigcup_{y\in\partial\cube_{n-1}}E_{y,k}\Big)
\end{align*}
hold true if each $E_{y,k}$ is a set satisfying $T_{y,k}\subset E_{y,k}\subset L_{y,k}$ for each $y\in\partial\cube_{n-1}$ and each $k\in\R$ with $|k|>1$. 

\vspace{2mm}
\begin{substeps}{box}
Next, for a fixed point $y\in\partial\cube_{n-1}$ we prove that 
$$
F_2(S_{y,k})=S_{y,k}\quad\text{if}\quad |k|\leq 1\quad\text{and}\quad T_{y,k}\subset F_2(S_{y,k})=E_{y,k}\subset L_{y,k}\quad\text{if}\quad |k|>1.
$$

Indeed, fix $k\in\R$ with $|k|\leq 1$ and consider the polynomial $h_k(\t)=g(\t y,k)\t\in\R[\t]$, which has degree $\geq1$ and satisfies that $h_k(1)=g(y,k)=1$, because $y\in\partial\cube_{n-1}$, and $h_{k}\big((1,+\infty)\big)\subset(1,+\infty)$, because $g(ty,k)\geq1$ (and so $h_k(t)>1$) whenever $t>1$ and $|k|\leq 1$. Hence, $h_{k}\big((1,+\infty)\big)=(1,+\infty)$ and therefore 
$$
F_2(S_{y,k})=\big\{\big(h_k(t)y,k\big)\in\R^n:\ t>1\big\}=\{(sy,k)\in\R^n:\ s>1\}=S_{y,k}.
$$
Now, fix $k\in\R$ with $|k|>1$ and consider again the univariate polynomial $h_k(\t)=g(\t y,k)\t\in\R[\t]$, which has degree $\geq1$ and negative leading coefficient. Thus, $\lim_{t\to+\infty}h_k(\t)=-\infty$. Moreover, $h_k(1)=g(y,k)=1$ and so $(-\infty,1)\subset h_k\big((1,\infty)\big)\subset\R$. Therefore, the set
$$
E_{y,k}=F_2(S_{y,k})=\big\{\big(h_k(t)y,k\big)\in\R^2:\ t>1\big\}=\big\{(sy,k)\in\R^n:\ s\in h_k\big((1,\infty)\big)\big\}
$$
satisfies $T_{y,k}\subset E_{y,k}\subset L_{y,k}$. Thus, we conclude that
\begin{multline*}
F_2\big(\R^n\setminus(\cube_{n-1}\times\R)\big)=\bigcup_{k\in\R}\ \bigcup_{y\in\partial\cube_{n-1}}F_2(S_{y,k})\\
=\Big(\bigcup_{|k|\leq1}\ \bigcup_{y\in\partial\cube_{n-1}}S_{y,k}\Big)\cup\Big(\bigcup_{|k|>1}\ \bigcup_{y\in\partial\cube_{n-1}}E_{y,k}\Big)=\R^n\setminus\cube_n,
\end{multline*}
as wanted.
\end{substeps}
\end{proof}

% COROLLARY
\begin{cor}\label{openball} 
The complement of an open ball in $\R^n$ is a polynomial image of $\R^n$ for $n\geq2$.
\end{cor}
\begin{proof}
Let us denote by ${\ball}_r$ the open ball of radius $r>0$ centered at the origin of $\R^n$. It suffices to show that $\R^n\setminus{\ball}_1$ is indeed a polynomial image of $\R^n$. For that, consider the $n$-dimensional cube $\cube_n=[-1,1]^n$ and recall that, by \ref{box}, there exists a polynomial map $F_1:\R^n\to\R^n$ whose image is $\R^n\setminus\cube_n$. Observe also that $\ball_1\subset\cube_n\subset\ball_n$.

\vspace{2mm}\setcounter{substep}{0}
\begin{substeps}{openball}
Let us construct now a polynomial $g\in\R[{\t}^2]$ such that the polynomial $h(\t)=\t g(\t)$ is an increasing function on the interval $[n,+\infty)$ and satisfies the following conditions: 
$$
h(0)=0,\ h(1)=h(n)=1,\ h'(n)=0,\ h\big((1,n)\big)\subset[1,+\infty)\quad\text{and}\quad h\big([0,+\infty)\big)=[0,+\infty).
$$ 
This implies, in particular, that $h\big((1,+\infty)\big)=h\big([n,\infty)\big)=[1,+\infty)$.

Let us check that the polynomial $g(\t)=a_n\t^4+b_n\t^2+c_n$, where the values of $a_n$, $b_n$ and $c_n$ are
respectively given by
$$
a_n=\dfrac{1+2n}{2n^3(1+n)^2},\quad
b_n=-\dfrac{1+2n+3n^2+4n^3}{2n^3(1+n)^2}\quad\text{and}\quad
c_n=\dfrac{3+6n+4n^2+2n^3}{2n(1+n)^2}
$$
does the job. Note that the values $a_n$, $b_n$, $c_n$ are obtained by imposing the conditions $h(1)=h(n)=1$ and $h'(n)=0$ on a polynomial of degree $5$ of the type $a_n\t^5+b_n\t^3+c_n\t$. Let us check that this polynomial satisfies all the other requested conditions. 
\end{substeps}

\vspace{2mm}
Indeed, it is clear that $h(0)=0$. If we write $m=n-1\geq1$, the derivative $\frac{dh}{d\t}=5a_n\t^4+3b_n\t^2+c_n$ of $h$ with respect to $\t$ at $t=1$ is
$$
\frac{n^4+n^3-4n^2+n+1}{(1+n)n^3}=\frac{m^2(m^2+5m+5)}{(2+m)(m+1)^3}>0,
$$
and so $h$ is increasing in a neighbourhood of $t=1$. Observe that for each $n\geq2$, we have $a_n,c_n>0$ and $b_n<0$. Thus, by Descartes' rule of signs (see for instance \cite[4-10, pag.158]{m}) the number of positive roots of $\frac{dh}{d\t}$ is upperly bounded by $2$. Applying Rolle's theorem to the equalities $h(1)=h(n)=1$, there exists $\xi\in(1,n)$ such that $\frac{dh}{d\t}(\xi)=0$. Since also $h'(n)=0$, the derivative $\frac{dh}{d\t}$ has exactly two positive roots, which are $\xi,n$. In particular, since $a_n>0$, we deduce that $\frac{dh}{d\t}$ is positive on $(n,+\infty)$ and so $h$ is an increasing function in the interval $[n,+\infty)$. 

Moreover, since $h$ is also increasing in a neighbourhood of $t=1$, $h(1)=h(n)=1$
and $\frac{dh}{d\t}$ has exactly one root in the interval $(1,n)$, we deduce that $h(t)>1$ for all $t\in(1,n)$. Otherwise, by Bolzano's theorem, there exists $t_0\in(1,n)$ such that $h(t_0)=1$ and, by Rolle's theorem, there exist $\xi_1\in(1,t_0)$ and $\xi_2\in(t_0,n)$ such that $\frac{dh}{d\t}(\xi_i)=0$ for $i=1,2$, a contradiction.

Even more, observe that $h$ is increasing in the interval $(0,\xi)$ (we have already checked that $\frac{dh}{d\t}(1)>0$). Since $h(\xi)>1=h(n)$ and $\frac{dh}{d\t}$ does not vanish on the interval $(\xi,n)$, the function $h$ decreases in such interval. Since we have already proved that $h$ is again an increasing function in the interval $[n,+\infty)$, the equality $h([0,+\infty))=[0,+\infty)$ follows.

\vspace{2mm}
\begin{substeps}{openball}
Next, since $g\in\R[{\t}^2]$, the map 
$$
F_2:\R^n\to\R^n,\ (x_1,\ldots,x_n)\mapsto g\Big(\sqrt{x_1^2+\cdots+x_n^2}\Big)(x_1,\ldots,x_n)
$$
is polynomial. For each closed ray $L_v^+=\{tv\in\R^n:\ t\geq0\}$ from the origin, where $v\in\R^n$ is a unitary vector, consider the polynomial function 
$$
[0,+\infty)\to[0,+\infty),\ t\mapsto\|F_2(tv)\|=g(t)t=h(t).
$$
Observe that for each unitary vector $v\in\R^n$ there exists a real number $\mu_v\in(1,n)$ such that
$$
L_v^+\cap(\R^n\setminus\cube_n)=\{tv\in\R^n:\ t\in[\mu_v,+\infty)\},
$$ 
and so $F_2\big(L_v^+\cap(\R^n\setminus\cube_n)\big)=\{tv\in\R^n:\ t\in[1,+\infty)\}$. Therefore, 
$$
F_2(\R^n\setminus\cube_n)=\bigcup_{\substack{v\in\R^n\\\|v\|=1}}F_2\big(L_v^+\cap(\R^n\setminus\cube_n)\big)=\bigcup_{\substack{v\in\R^n\\\|v\|=1}}\big\{tv\in\R^n:\ t\in[1,+\infty)\big\}=\R^n\setminus{\ball}_1.
$$
\end{substeps}

Hence, the polynomial map $F=F_2\circ F_1:\R^n\to\R^n$ satisfies $F(\R^n)=\R^n\setminus{\ball}_1$.
\end{proof}

It was also proved in \cite[4.2]{fg2} (resp. \cite[ 6.2]{fgu1}) that the complement of a closed disc in $\R^2$ (resp. of a closed ball in $\R^3$) is a polynomial image of $\R^3$ (resp. of $\R^4$). To finish, we present the $n$-dimensional version of this result following the same strategy already presented there.

\begin{cor}\label{closed ball}
The complement of a closed ball in $\R^n$ is a polynomial image of $\R^{n+1}$.
\end{cor}
\begin{proof}
It is enough to prove that $\R^n\setminus\ol{\ball}$, where $\ol{\ball}$ is the closed ball in $\R^n$ of radius 1 centered at the origin, is indeed a polynomial image of $\R^{n+1}$. By \ref{openball}, there exists a polynomial map $G_1:\R^n\to\R^n$ whose image is $\R^n\setminus{\ball}$, where ${\ball}$ denotes the open ball in $\R^n$ of center the origin and radius $1$. On the other hand, by \ref{step1} (see also \cite[1.4(iv)]{fg1}), the image of the polynomial map
$$
G_2:\R^2\to\R^2, \ (x,y)\mapsto((xy-1)^2+x^2,y(xy-1))
$$
is the open half-plane $\{(x,y)\in\R^2:\ x>0\}=(0,+\infty)\times\R$. Consider the polynomial maps
$$
\begin{array}{rl}
F_1:\R^{n+1}\to\R^n,&\ (x_1,\ldots,x_{n+1})\mapsto (1+x_1)G_1(x_2,\ldots,x_{n+1}),\\[4pt]
F_2:\R^{n+1}\to\R^{n+1},&\ (x_1,\ldots,x_{n+1})\mapsto (G_2(x_1,x_2),x_3,\ldots,x_{n+1})
\end{array}
$$
Observe that $F_2(\R^{n+1})=(0,+\infty)\times\R^n$ and so $(F_1\circ F_2)(\R^{n+1})=F_1\big((0,+\infty)\times\R^n\big)=\R^n\setminus\ol{\ball}$, as wanted.
\end{proof}


% THE BIBLIOGRAPHY
\begin{thebibliography}{FGU}

\bibitem[BCR]{bcr} J. Bochnak, M. Coste, M.F. Roy: Real algebraic geometry. {\em Ergeb. Math.} {\bf 36}, Springer-Verlag, Berlin: 1998.

\bibitem[FG1]{fg1} J.F. Fernando, J.M. Gamboa: Polynomial images of $\R^n$. \em J. Pure Appl. Algebra \em {\bf 179}, no. 3, (2003) 241--254.

\bibitem[FG2]{fg2} J.F.Fernando, J.M. Gamboa: Polynomial and regular images of $\R^n$. \em Israel J. Math. \em {\bf 153}, (2006) 61--92.

\bibitem[FGU]{fgu1} J.F. Fernando, J.M. Gamboa, C. Ueno: On complements of convex polyhedra as polynomial images of $\R^3$. \em Preprint RAAG\em, Madrid-Fuerteventura: 2009.

\bibitem[G]{g} J.M. Gamboa: Reelle Algebraische Geometrie, June,
$10^{\text{th}}-16^{\text{th}}$ (1990), Oberwolfach.

\bibitem[M]{m} B. E. Meserve: Fundamental Concepts of Algebra, Dover Publications, New York: 1982.

\bibitem[U1]{u1} C. Ueno: A note on boundaries of open polynomial images of $\R^2$. \em Rev. Mat. Iberoamericana\em, {\bf 24}, no. 3, (2008) 981--988.

\bibitem[U2]{u2} C. Ueno: On convex polygons and their complements as images of regular and polynomial maps of $\R^2$. \em Preprint \em RAAG, Fuerteventura: 2009.

\end{thebibliography}
\end{document}